\chapterimage{./shell/cover1} % Chapter heading image
\chapter{Shell - TP1 : Manipulations de l’environnement et des fichiers sous UNIX}
    \section{Exercice 1 : Découverte de quelques commandes d'archivage}
    \textit{L'objectif de cet exercice est de découvrir et manipuler les commandes de téléchargement, d'archivage, de compression et de décompression de fichier}
        \subsection{1. Récupération et décompression d'une archive}
            La commande \texttt{wget https://cloud.infotro.fr/ITC313/archive.tar} permet de télécharger l'archive présent à cette adresse.
            \begin{itemize}
                \item L'option \texttt{-x} permet de restaurer les fichiers contenus dans une archive.
                \item L'option \texttt{-c} permet de créer une nouvelle archive.
                \item L'option \texttt{-f} permet d'utilise le fichier archive F ou le périphérique F (par défaut /dev/rmt0).
            \end{itemize}
            9 fichiers était présent dans cette archive
        \subsection{2. Manipulation de fichiers}
            \texttt{file .\/*}
            Afin de renommer le fichier j'utilise cette commande \texttt{mv image4.jpg image4.jpg2}.
            Le fichier \texttt{script.txt} fait 170Ko.
            La commande \texttt{gzip} sur \texttt{script.txt} a compressé le fichier.
            Le fichier fait maintenant  65Ko.
            La compréssion est donc d'environ 38.235\%.
            Après décompression avec la commande \texttt{gunzip} le fichier fait maintenant 170Ko, qui est la taille initial du fichier.
        \subsection{3. Création d'une nouvelle archive}
            La nouvelle archive fait 850Ko soit 1Ko de moins que l'ancienne archive. Surement à cause du nom de fichier jp2 changé.
            La somme des tailles des fichiers dans l'archive est égale à 845479 soit 845Ko, on observe une différence de 5Ko.
            La compression de l'archive (créé précédemment) fait 617Ko soit une différence de 228Ko.
            L'option \texttt{-z} utilise gzip pour comprésser l'archive.
            Elle revient totalement à créé une archive puis la compresser puisque d'après les test la taille n'est pas différente.
            La Commande \texttt{tar -c -z *.jpg *.txt *.jp2} devrait normalement afficher dans le terminal le résultat.
            La Commande \texttt{tar -c -z *.jpg *.txt *.jp2 > nouvelleArchive3.tar.gz} redirige bien le résultat dans un fichier.
            La redirection du flux dans un fichier recréer une archive compréssé similaire à la deuxième créé.
            En conclusion l'archive 2 et 3 donne le même résultat et sont plus petit que l'archive 1 puisqu'elles sont compréssés.
    \section{Exercice 2 : Utilisation des masques de création de fichiers}
        \subsection{1.}
            \begin{minted}[linenos]{shell}
                $ touch Raphael.txt
                $ umask 0666
                $ touch Donatello.txt
                $ umask 0331
                $ touch Michelangelo.txt
                $ umask 0661
                $ touch Leonardo.txt
                $ umask 0000
            \end{minted}
        \subsection{2. et 3.}
            Il n'est pas possible de créer de donner plus de droit que la limitation par défaut du systeme.
            ( application umask par défaut 666 sur fichier et 777 sur repertoire)
    \section{Exercice 3 : Manipulation du Systeme de fichier et des droits de navigation}
        \subsection{1.}
            L'archive contient 5 images.
        \subsection{3.}
            \texttt{/home/ESIREM-AD/al669724/Documents/Shell/TPs/TP1/Ex3/images/Chinpokomon/P-Z/Vamporc.png}
        \subsection{4.}
            \texttt{../P-Z/Vamporc.png}
        \subsection{6.}
            L'option \texttt{-z} permet de compresser l'archive.
            \begin{verbatim}
                $ tar -xczf ITC313_TP_Shell_lebot.axel.tar.gz
            \end{verbatim}
            permettra de décompresser et extraire l'archive.
        \subsection{7.}
            Toutes les permissions sont conservés.
    \section{Exercice 4 : Manipulation d'expression régulière}
        \subsection{2.}
            Les lignes contenant la suite de lettres "ette".
        \subsection{3.}
            Les lignes contenant la lettre "T".
        \subsection{4.}
            Les lignes commencant par la lettre "T".
        \subsection{5.}
            \texttt{\^} signifie "début".
        \subsection{6.}
            Les lignes finissant par "te".
        \subsection{7.}
            Les lignes contenant la suite de caractère "c", un caractère, "r".
        \subsection{8.}
            Les lignes contenant "oui" ou "non".
        \subsection{9.}
            \begin{itemize}
                \item '\$' -> en fin
                \item '|' -> ou
                \item '.' -> un caractère
            \end{itemize}
        \subsection{10.}
            Permet d'afficher uniauement la partie correspondant à la recherche.
        \subsection{11.}
           Une suite de 4 lettre Majuscule.
        \subsection{12.}
            Une suite d'au moins une Majuscule et un minuscule
        \subsection{13.}
            Les mots commencant par une majuscule aisni que les lettre majuscules.
        \subsection{15.}
            Permet de récupérer les addresse e-mail.
        \subsection{16.}
            Permet de récupérer les numéro de téléphone.
        \subsection{17.}
            ((bien))((joue))((tu))((as))((trouve))((la))((reponse))((a))((la))((derniere))((question))
