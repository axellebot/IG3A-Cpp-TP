\chapter{TP7 + TP8 programmation : communication socket}
    \section{Communication distante en utilisant l’outil netcat}
        \subsection{Exercice 1 : Découverte de la commande nc : netcat}
        \subsection{Exercice 2 : Utilisation de la commande nc : netcat pour le transfert de fichier et l’évaluation de
la bande passante}
        \subsection{Exercice 3 : Une histoire de serveurs concurrents ...}
        \subsection{Exercice 4 : Comprendre une requête HTTP}
    \section{Développement d’un client et d’un serveur en C}
        \subsection{Exercice 5 : Mise en place d’une communication en mode non connecte}
        \subsection{Exercice 6 : Création d’une architecture (client UDP) - (relai UDP-TCP)- (serveur TCP)}
    \section{Exercices bonus}
        \subsection{Exercice 7 : Résolution de noms}
        \subsection{Exercice 8 : Serveur multi-client en mode connecte}
